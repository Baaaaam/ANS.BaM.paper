\section{Motivation}

Gas centrifuge cascades are usually designed to operate in ideal manner, with no
losses in separative work, the most effective way. To achieve such ideal
configuration, the cascade is designed to be fed with a specific feed assay and
produce the target  enrichment while rejecting tails at a fix assay.

With the current international tensions regarding enrichment capabilities, this
work aims to measure the effectiveness of an enrichment cascade when used outside of
its designed scope and quantify the attractiveness of such way to build up
significant amount of \gls{HEU}.

The present work investigates the performance of a enrichment cycle when chaining
gas enrichment cascade tuned for low enrich uranium production from natural
uranium. As literature on the mater is for obvious reason limited, three
behavior models have been implemented and used to evaluate the response of an
enrich cascade when fed with different assays than the design one. This work
takes also advantage of the Cyclus\cite{cyclus} fuel cycle capabilities to
evaluate the assay blending equilibrium.

