\section{Theory}
\subsection{Centrifuge properties}

The present work uses the analytical solution by R\"aetz \cite{raetz.phd} of the
differential equation for the gas centrifuge as described in \cite{glaser.2008}.
Centrifuge parameters, such as average gas temperature, $T$, peripheral speed,
$v$, height, $h$, diameter, $d$, pressure ratio, $x$, feed flow rate, $F$,
counter-current flow ratio, $L/P$, and efficiency, $e$ have been chosen (Table
\ref{tab:centrifuges}) to match the cascade design describe in
\cite{glaser.2008} and \cite{walker.2017} using P1-type centrifuges.

\begin{table}[htb]
\centering
\caption{Summary of the centrifuge parameters.}
\begin{tabular}{cccccccc}
\toprule
$T$[K] & $v$[m/s]    & $h$[m] & $d$[m]   & $x$   & $F$[mg/s]  & $L/F$ & $e$  \\
\midrule
320    & $320$           & $1.8$ & $0.105$ & $1e3$  & $13$      & 2     & 1.0  \\
\bottomrule
\end{tabular}

  \label{tab:centrifuges}
\end{table}

\subsection{Cascade Design}

The cascade is built as an ideal cascade, with no losses in the separative work,
which corresponds to $\alpha =\beta = const$ for all stages of the cascade, where
$\alpha$ and $\beta$ respectively represent the feed to product and the feed to
tail enrichment factors.  $\alpha$ and $\beta$ can be expressed as function of
the abundance ($R$) or the enrichment ($N$) of respectively the product
($R'$, $N'$) and the feed ($R$, $N$) and the feed and the tails ($R''$, $N''$) such
as:
\begin{subequations} \label{eqs:alphabeta}
    \begin{equation} \label{eq:alpha_def}
        \alpha = \frac{R'}{R} = \frac{N'}{1-N'}\frac{1-N}{N} 
\end{equation}
\\
\begin{equation}\label{eq:beta-def}
        \beta = \frac{R}{R''} = \frac{N}{1-N}\frac{1-N''}{N''} 
\end{equation}
\end{subequations}

As detailed in \cite{avery} it is also possible to derive $\alpha$ from the
first principle, and express it as a function of the feed rate $F$, the
separative performance $\delta U(\theta)$ and the cut $\theta$:

    \begin{equation} \label{eq:alpha}
    \alpha = \sqrt{\frac{2\delta U}{F} \frac{1-\theta}{\theta}}+1
\end{equation}

From the mass conservation, $N = \theta N' + (1-\theta)N''$, and equations
\eqref{eqs:alphabeta} it is possible to express $\beta$ as a function of the
feed abundance, $R$, the cut $\theta$ and $\alpha$:

\begin{equation}\label{eq:beta}
    \beta =   R \left(\dfrac{1-\theta}
                     {\dfrac{R}{R+1}- \theta \dfrac{\alpha R}{1+\alpha R}} -1\right)
\end{equation}


From equation \eqref{eq:alpha} and \eqref{eq:beta} it is possible to determine
the cut, $\theta$, or the ratio of product flow to feed flow required to build an ideal
cascade:
$\beta$ values and the feed assay, $N_{i}$:
\begin{eqnarray}
    \theta_{i} = \dfrac{N_{i} - \dfrac{1}{1 + \beta/R_{i}}}{ \dfrac{\alpha R_{i}}{1 + \alpha R_{i}} -
           \dfrac{1}{1 + \beta/R_{i}}}
\end{eqnarray}

As $\alpha_{i}$ and $\beta_{i}$ remain constant, only the value of the cut,
$\theta_{i}$, changes across the different stages of a cascade.  This algorithm
assumes that the corresponding separative power $\delta U$ (not re-computed) can
be achieved with the chosen centrifuge design, tuning other operational
parameter such as the rotation speed, the counter-current flow ratio...  Once
$\theta_{i}$ is determined, it is possible to compute the product and the tail
assay.

The design of the cascade is performed through 2 steps.  First one determines
the configuration and number of stages, adding stages until the product assay of
the final stage is greater or equal the product targeted assay, and similarly
the tails assay is less or equal the tails desired assay.  This determines the
number of enriching and stripping stages as well as their enrichment properties
($N_{i}$, $N'_{i}$, $N''_{i}$,$\theta_{i}$i).


The second step determines the relative flows at each stages, solving the linear
flow equation, \eqref{eq:flow}.
The cascade can then be populated with actual machines until the maximum number
available of machines is reached.

\begin{equation}
\setcounter{MaxMatrixCols}{20}
\begin{bmatrix}
     -1        & 1-\theta_{_{S+1}} & 0 & ...  & 0              & 0              & 0                 & 0                 & 0                 & ...  & 0               & 0 \\
 \theta_{_{S}} & -1                & 0 & ...  & 0              & 0              & 0                 & 0                 & 0                 & ...  & 0               & 0 \\
               &                   &   &     &                &                & ...               &                   &                   &     &                 &   \\
 0             & 0                 & 0 & ...  & \theta_{_{-2}} & -1             & 1 - \theta_{_{0}} & 0                 & 0                 & ...  & 0               & 0 \\
 0             & 0                 & 0 & ...  & 0              & \theta_{_{-1}} & -1                & 1 - \theta_{_{1}} & 0                 & ...  & 0               & 0 \\
 0             & 0                 & 0 & ...  & 0              & 0              & \theta_{_{0}}     & -1                & 1 - \theta_{_{2}} & 0   & ...             & 0 \\
               &                   &   &     &                &                & ...               &                   &                   &     &                 &   \\
 0             & 0                 & 0 & ...  & 0              & 0              & 0                 & 0                 & 0                 & ...  & -1              & 1-\theta_{_{E}} \\
 0             & 0                 & 0 & ...  & 0              & 0              & 0                 & 0                 & 0                 & ...  & \theta_{_{E-1}} & -1
 \end{bmatrix}
 \times
 \begin{bmatrix}
     F_{_{S}}   \\
     F_{_{S+1}} \\
     \cdots     \\
     F_{_{-1}}  \\
     F_{_{0} }  \\
     F_{_{1} }  \\
     \cdots     \\
     F_{_{E-1}} \\
     F_{_{E}}
 \end{bmatrix}
 =
 \begin{bmatrix}
     0      \\
     0      \\
     \cdots \\
     0      \\
     F      \\
     0      \\
     \cdots \\
     0      \\
     0
\end{bmatrix}
%\caption{caption needed!}
\label{eq:flow}
\end{equation}



\subsection{Miss-use models}

Little information is available about optimising an existing enrichment cascade
that is being fed with a feed enrichment that does not match the design one. So
far 3 different methods have been investigated. 

The first one assumes that no change are been made on the cascade, i.e $\delta
U$, $F$ and $\theta$ are fixed across all stages. The second one assumes the cut
value at each stage is retuned to maintain the ideal state of the cascade,
$\alpha$ and $\beta$ remain fixed. The last one, described in \cite{walker.2017}
assumes the tails to product enriching factor and the cut remain constants ($\gamma =
\alpha\times\beta$). Models behaviors and assumptions are summarized in Tab.
\ref{tab:models}.

\begin{table}[htb]
\centering
  \caption{Summary of miss-use model properties.}
\begin{tabular}{l|ccc}
\toprule

Model                &    A                 & B                  & C  \\
\midrule
Constant parameters  & $\alpha_i, \theta_i$ & $\alpha_i=\beta_i$ & $\gamma_i=\alpha_i*\beta_i, \theta_i$       \\
Optimised parameters & $\beta_i$            & $\theta_i$         & $\alpha_i, \beta_i$                     \\
Assays determination & blended              & ideal              & blended                  \\
Flow                 & unchanged            & scaled             & unchanged       \\

\bottomrule
\end{tabular}
  \label{tab:models}
\end{table}


\subsubsection{Model A}

The tuning method A does not re-optimize $\theta_i$ keeping the same flow as the
ideal configuration. From equation \eqref{eq:alpha}, maintaining $\delta U$, $F$
and $\theta$ unchanged implies $\alpha$ remains unchanged as well. According to
equation \eqref{eq:beta}, when $\alpha$ and $\theta$ are fixed, if the feed
assay ($N$) changes, $\beta$ will change accordingly.  This breaks the
ideal status of the cascade, i.e. $N_{i} \neq N'_{i-1} \neq N''_{i+1}$.

In order to compute the proper product and tails assay at each stage, the tails
and the product from respectively the next and the previous stage must be
blended in order to determine the correct stage feed assay. As this is a obvious
cycling problem, an iterative solution has been chosen: all feed assays are
iteratively updated, blending the proper product and tails, then using the
updated feed assay, the new product and tail assays are recomputed. This process
is repeated until the sum of the square difference in assays is smaller than the
precision (1e-8).  As the cut remain fixed at each stage the different flow do
not need to be recomputed.


\subsubsection{Model B}

The second method the cut value at each stage $\theta_i$, is retuned in order to
maintain the $\alpha_i$ and $\beta_i$ at their original values. As the cascade
remains ideal, the product and tail assay at each stages (and for the overall
cascade) is easily determined using equations \eqref{eqs:alphabeta}.

As the cuts values change, the flow rate between the different stage has to be
recomputed. Because the cascade is not reorganised (the number of cascade per
stage remain the same as the original design). The new flow rate are computed as
the flow rates of the reconfigured cascade scaled down to match the allowed
flows of the original one: the flow rates at each stage of a miss-used cascade
have to be inferior or equal to the initial cascade design.

\subsubsection{Model C}
The last model assumes that the tail to product enrichment factor remains
constant regardless to the feed assays. To compute the response of the cascade
one need to determine $\alpha$ and $\beta$ such as their product and
$\theta$ remain fixed. 
From equations \eqref{eqs:alphabeta} and the assay conservation equation $N =
\theta N' + (1-\theta)N''$ it is possible to express the product $N'$ as a function of
the feed assay $N$, $\gamma$ and the cut $\theta$ as one solution of the second
order equation \eqref{eq:gamma_p}:

\begin{equation}\label{eq:gamma_p}
\gamma(1-N')(P\theta-N) - N'(1-(N'\theta-N)/(-1+\theta))(-1+\theta) = 0
\end{equation}

The only solution allowing product assay to range between 0 and 1 is the
following :
\begin{equation}\label{eq:model_b_sol}
    \frac{\gamma(N + \theta) - N 
            - \sqrt{\gamma^2(N^2 
                            - 2N\theta 
                            + \theta^2) 
                    - \gamma(2N^2 - 2\theta^2 
                            + 2N + 2\theta) 
                    + N^2 + 2N\theta + \theta^2 - 2N - 2\theta + 1} 
            - \theta + 1}
        {2\theta(\gamma - 1)}
\end{equation}
Once the product assay is known, one can trivially determine the tail assay,
$\alpha$ and $\beta$ using equations \eqref{eqs:alphabeta} and mass
conservation.

Similarly as model A, because the cut values remain constant, the flows don't
need to be recomputed, and the correct assays, $\alpha$ and $\beta$ are
determined through iterative blending of the product assays of the previous
stage and the tails assay of the next stage using equation
\eqref{eq:model_b_sol}.


