\section{Discussion: Limitations of the Framework}
\label{s_discuss}
The use of historical data to develop a forward model has several limitations. The biggest limitation is in the size of this data set, both in terms of the number of states and in the range of years that are included.  With only 10 states that have acquired nuclear weapons in the historical record, there is large uncertainty in any quantitative analysis. As a result, while the analysis has identified factors that are correlated to weapons programs, there is insufficient data to confirm causal relationships.

Since the data set already includes all states that have pursued nuclear weapons, the missing states would only be relevant to the likelihood analysis if they had pursuit scores of 4 or larger, thus reducing the time-integrated likelihood of pursuit associated with such scores. Due to the influence of nuclear technology, a non-nuclear state could theoretically have a maximum pursuit score of 7. Considering that the majority of states that have high potential military spending and major historical conflicts have already been incorporated, this further reduces the potential maximum score to below 5, even if all other factors are maximized.  A large fraction of states in the world are expected to have scores on the order of 1-3, and thus not alter the distribution shown in \ref{fig:likely}.

A more important addition to the database would be data for each state for all years between 1940 (before the first historical pursuit) and the year of pursuit and/or aquisition.  This would allow the direct calculation of an annual probability of pursuit as a function of the pursuit score.  The current model for converting a time-integrated likelihood to an annual probability inherently assumes that the pursuit score for the states in the data set is constant for all the years prior to the decision to pursue nuclear weapons.  An effort to assemble this data would be a valuable contribution to this modeling effort.



