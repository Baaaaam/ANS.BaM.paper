\section{Discussion}

We can observe that when the cascade is left completely untouched (Model A) or
when it is slightly retuned to maintain the tails to product enrichment factor
as well as the cut of each centrifuges (Model C), chaining the cascade can
achieve large increase of the enrichment at each level.  On the contrary, when
retuning the cut of each centrifuges to maintain the ideal state of the cascades
(Model B) while chaining them, the \gls{HEU} production rate is favored over the
enrichment gain.

The tails recycling allows each model to achieve a large gain in productivity,
even for then model B in which the number of levels required to reach $90w\%$ of
$^{235}$U in the uranium does not change. Even if no cascade chaining options
achieves the same production rate as direct enrichment, the model B with tail
recycling reached about $80\%$ of the optimum production rate. Such production
rate would allow the accumulation of a Significant Quantity of \gls{HEU} in less
than 8 months\ldots


\section{Conclusion and future work}

This work has investigated the possibility to chain centrifuge enrichment
cascades that are designed to enrich uranium for commercial reactors in order to
produce \gls{HEU}, instead. Three methods have been implemented to model
symmetric enrichment cascade behavior when fed with different uranium enrichment
than the designed enrichment. 

%% This work has investigated and quantified the difference between potential
%% models for retuning of a centrifuge enrichment cascade in order to chain them to produce
%% \gls{HEU} initially tuned to produce uranium enrichment for commercial reactors.
One of these method achieves up to $80\%$ of the production rate of a single
large enrichment cascade designed specifically for \gls{HEU} production using
the same number of centrifuges.

This work will be extended to the near future with additional misuse methods,
allowing for example, the reconfiguration of the centrifuges in the cascades.

For this study, the use of the Cyclus fuel cycle simulator was not required;
it only allows a quick determination of the blending equilibrium. Future
studies will make use of the full capability of Cyclus Dynamic Resource
Exchange in order to automatically assign the different cascades to the
different level as function of the resources availability, optimising the
productions rates in each cases.

While mathematically correct, the authors do not guaranty the feasibility of the
different misuse tuning methods implemented and are welcoming any insight on the
matter.


